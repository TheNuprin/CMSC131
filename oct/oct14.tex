\section{Monday, October 14, 2019}

Today, we'll start talking about arrays.

\subsection{Arrays}

Suppose we have a \verb!Student! class, and we want to keep track of three students. From what we've learned so far, we could just initialize three \verb!Student! variables. But what happens if we need to keep track of $100$ students? Or $10,000$ students? Clearly, our current method is not feasible. This is where arrays come into picture. \\

An \vocab{array} in Java is an object used to store a fixed-size sequential collection of elements of the same type. The benefits of an array is that we can treat an array as a single entity, but we can also access each of the individual elements that are stored inside of the array. \\

How do we declare an array? We can declare an array of a given type with the general format:

\[
\verb!type[] variablename;!
\]

For example, we can declare an array of \verb!int!s named \verb!arr! by simply writing \verb!int[] arr;!. Similarly, we can create an array of characters by writing \verb!char[] arr;!. \\

How do we initialize an array? We use the \verb!new! keyword, we respecify the type, and we finally specify the size of the array. The general syntax is provided below:

\[
\verb!type[] variablename = new type[size];!
\]

For example, if we want to declare an array of $10$ integers, then we can write,

\[
\verb!int[] arr = new int[5];!
\]

\noindent The code above would initialize an array of $5$ \verb!int!s, and we would now have $5$ \verb!int!s at our disposal. By default, the values of an \verb!int! array are all initialized to $0$. We view an array as a a contiguous block in memory, so we might visualize \verb!arr! as follows:

\[
\fbox{0} \sep \fbox{0} \sep \fbox{0} \sep \fbox{0} \sep \fbox{0}
\]

\noindent We can access each of the individual \verb!int! values in our array by writing the name of the array, followed by square brackets, with a number corresponding to the entry that we wish to access (starting from $0$). For example, we can change the first value in our array by writing \verb!arr[0] = 5;!, and we can change the last element in our array as \verb!arr[4] = 7;!. Afterwards, our array will look like the following:

\[
\fbox{5} \sep \fbox{0} \sep \fbox{0} \sep \fbox{0} \sep \fbox{7}
\]

\noindent The process of accessing an individual element in an array is called \vocab{indexing}. 

Here's some code illustrating some of the nice things that we can do with arrays:

\begin{lstlisting}
public class ReadValues {

	public static void main(String[] args) {
		Scanner scanner = new Scanner(System.in);

		System.out.println("How many values would you like to store: ");
		int size = scanner.nextInt();

		int[] nums = new int[size];

		for (int i = 0; i < nums.length; i++) {
			System.out.println("Enter value " + (i + 1) + ": ");
			nums[i] = scanner.nextInt();
		}

		System.out.println("Here are the values you entered:  ");
		for (int i = 0; i < nums.length; i++) {
			System.out.println(nums[i]);
		}

		scanner.close();
	}
}
\end{lstlisting}

A summary of what's going on in this code is presented below:

\begin{itemize}
    \item Firstly, we ask our user to enter the number of values they want to store. We read in this value, and we initialize an array with that size.  
    \item Next, we read in the corresponding number of values by using a for-loop. In each iteration of the for-loop, we ask the user to enter the next value, and we store it into the next entry in the array. Note that our array has a built-in \verb!.length! field that we can access. This is simply an integer variable that tells us the size of the array.
    \item Finally, we use another for-loop to iterate over our array. In the \verb!i!th iteration of the for-loop, we print \verb!nums[i]!, which is the \verb!ith! element in our array. 
\end{itemize}

It's important to note that arrays start at $0$. This means that, if we're storing $10$ elements, then it's only valid to access the array at the indices $0, 1, 2, \ldots, 9$. What happens if we try to access an element that's out of bounds? An exception is thrown, and our Java program will abort if we don't have a \verb!try-catch! block. 


To summarize, there are three different situations that we've learned so far in which we can use the square brackets \verb![..]!. 

\begin{enumerate}
    \item Firstly, we use square brackets when we're trying to declare an array (e.g. \verb!int[] arr;!).
    \item We also use square brackets when we're initializing an array (i.e. \verb!arr = new int[10];)!.
    \item Finally, we also use square brackets when indexing an array (i.e. \verb!arr[0]! accesses the first element, \verb!arr[1]! accesses the second element, and so on).
\end{enumerate}

\subsection{Copying Arrays}

Consider the following code segment:

\begin{lstlisting}
public class Example {
    public static void main(String args[]) {
        int[] a = new int[5];
        for (int i = 0; i < 5; i++) {
            a[i] = i;    
        }
        int[] b = a;
    }
}
\end{lstlisting}

First, we initialize a variable \verb!a!. Subsequently, we use a for-loop, and we set \verb!a[i]! equal to \verb!i! for $i = 0, 1, 2, 3, 4$.  Next, we create a new variable called \verb!b!, and we set it equal to \verb!a!. 

Would this copy the elements of \verb!a! into \verb!b! (does \verb!b! contain the elements $0, 1, 2, 3, 4$)? The answer is no --- this just makes \verb!a! and \verb!b! aliases for each other. This is similar to creating a \verb!String! variable, and setting a new \verb!String! variable equal to it. \\

So how do we copy the contents of \verb!a! into \verb!b!? We need to use a for-loop and iterate over the elements of \verb!a!, while assigning the \verb!i!th element of \verb!a! to \verb!b[i]!. This is shown below:

\begin{lstlisting}
public class Example {
    public static void main(String args[]) {
        int[] a = new int[5];
        
        for (int i = 0; i < 5; i++) {
            a[i] = i;
        }
        
        int[] b = new int[5];
        /* Copy contents of a into b. */
        for (int i = 0; i < a.length; i++) {
            b[i] = a[i];
        }
    }
}
\end{lstlisting}